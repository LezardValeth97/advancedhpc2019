\documentclass{article}
\usepackage[utf8]{inputenc}

\title{Report.8.scatter.tex}
\author{Kieu Quoc Viet }
\date{November 2019}

\begin{document}

\maketitle

\begin{enumerate}
    \item Explain how you implement the labworks
    
    \begin{verbatim}
// Declare the SoA
struct hsv {
    float *h, *s, *v;
};

// Write a kernel
__global__ void RGB2HSV(uchar3* input, hsv output, 
        int imageWidth, int imageHeight){
    int tidx = threadIdx.x + blockIdx.x * blockDim.x;
    int tidy = threadIdx.y + blockIdx.y * blockDim.y;
    if(tidx >= imageWidth || tidy >= imageHeight) return;
    int tid = tidx + tidy * imageWidth;

    float r = (float)input[tid].x/255.0;
    float g = (float)input[tid].y/255.0;
    float b = (float)input[tid].z/255.0;

    float Max = max(r, max(g,b));
    float Min = min(r, min(g,b));
    float delta = Max - Min;

    float h = 0;
    float s = 0;
    float v = 0;

    if (Max != 0){
        s = delta/Max;
        if (Max == r) h = 60 * fmodf(((g-b)/delta),6.0);
        if (Max == g) h = 60 * ((b-r)/delta+2);
        if (Max == b) h = 60 * ((r-g)/delta+4);
    }

    if (Max == 0) s = 0;
    if (delta == 0) h = 0;
    v = Max;

    output.h[tid] = h;
    output.s[tid] = s;
    output.v[tid] = v;
}

__global__ void HSV2RGB(hsv input, uchar3* output, 
        int imageWidth, int imageHeight){
    int tidx = threadIdx.x + blockIdx.x * blockDim.x;
    int tidy = threadIdx.y + blockIdx.y * blockDim.y;
    if(tidx >= imageWidth || tidy >= imageHeight) return;
    int tid = tidx + tidy * imageWidth;
    
    float h = input.h[tid];
    float s = input.s[tid];
    float v = input.v[tid];

    float d = h/60;
    float hi = (int)d % 6;
    float f = d - hi;
    float l = v * (1-s);
    float m = v * (1-f*s);
    float n = v * (1-(1-f)*s);

    float r,g,b;
    if (h >= 0 && h < 60){
        r = v;
        g = n;
        b = l;
    }

    if (h >= 60 && h < 120){
        r = m;
        g = v;
        b = l;
    }

    if (h >= 120 && h < 180){
        r = l;
        g = v;
        b = n;
    }

    if (h >= 180 && h < 240){
        r = l;
        g = m;
        b = v;
    }

    if (h >= 240 && h < 300){
        r = n;
        g = l;
        b = v;
    }

    if (h >= 300 && h < 360){
        r = v;
        g = l;
        b = m;
    }

    output[tid].x = r*255;
    output[tid].y = g*255;
    output[tid].z = b*255;
}


void Labwork::labwork8_GPU() {
    // Calculate number of pixels
    int pixelCount = inputImage->width * inputImage->height;

    // // Allocate CUDA memory
    uchar3 *devInput;
    uchar3 *devOutput;
    hsv devHSV;
    cudaMalloc((void**)&devHSV.h, pixelCount *sizeof(float));
    cudaMalloc((void**)&devHSV.s, pixelCount *sizeof(float));
    cudaMalloc((void**)&devHSV.v, pixelCount *sizeof(float));
    cudaMalloc(&devInput, pixelCount *sizeof(uchar3));
    cudaMalloc(&devOutput, pixelCount *sizeof(uchar3));

    // // Copy InputImage from CPU (host) to GPU (device)
    cudaMemcpy(devInput, inputImage->buffer, 
    pixelCount * sizeof(uchar3),cudaMemcpyHostToDevice);

    // Processing : launch the kernel
    // int blockSize = 1024;
    // int numBlock = pixelCount / blockSize;  
    // grayscale<<<numBlock, blockSize>>>(devInput, devOutput);
    dim3 blockSize = dim3(32, 32);
    // //dim3 gridSize = dim3(8, 8);
    dim3 gridSize = dim3((inputImage->width + blockSize.x -1) / 
    blockSize.x, (inputImage->height + blockSize.y -1) / blockSize.y);
    // grayscaleVer2D<<<gridSize, blockSize>>>(devInput, 
    devOutput, inputImage->width, inputImage->height);
    RGB2HSV<<<gridSize, blockSize>>>(devInput, devHSV, 
    inputImage->width, inputImage->height);
    HSV2RGB<<<gridSize, blockSize>>>(devHSV, devOutput, 
    inputImage->width, inputImage->height);

    // Copy CUDA Memory from GPU to CPU
    // allocate memory for the output on the host
    outputImage = static_cast<char *>(malloc
        (pixelCount * sizeof(uchar3)));  
    cudaMemcpy(outputImage, devOutput, pixelCount * sizeof(uchar3),
        cudaMemcpyDeviceToHost);

    // Cleaning
    cudaFree(devInput);
    cudaFree(devOutput);
    cudaFree(devHSV.h);
    cudaFree(devHSV.s);
    cudaFree(devHSV.v);    
}
\end{verbatim}

\end{enumerate}

\end{document}
        


