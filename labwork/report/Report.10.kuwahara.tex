\documentclass{article}
\usepackage[utf8]{inputenc}

\title{Report.10.kuwahara.tex}
\author{Kieu Quoc Viet }
\date{November 2019}

\begin{document}

\maketitle

\begin{enumerate}
    \item Explain how you implement the labworks

    \begin{verbatim}
// Write a kernel
__global__ void kuwahara(uchar3* input, uchar3* output, 
        int imageWidth, int imageHeight, int windowSize){
    unsigned int tidx = threadIdx.x + blockIdx.x * blockDim.x;
    unsigned int tidy = threadIdx.y + blockIdx.y * blockDim.y;
    if(tidx >= imageWidth || tidy >= imageHeight) return;
    int tid = tidx + tidy * imageWidth;

    double window[4] = {0.0};
    double SD[4] = {0.0};
    int meanRGB[4][3] = {0};
    int pxCount[4] = {0};
    int windowPosition;

    for(int x = 1 - windowSize; x <= windowSize - 1; x++){
        for(int y = 1 - windowSize; y <= windowSize - 1; y++){
            int rows = tidx + x;
            int columns = tidy + y;
            if( rows < 0 || rows >= imageWidth || columns < 0 
                || columns >= imageHeight) continue;
            int positionOut = rows + columns * imageWidth;

            int red = input[positionOut].x;
            int green = input[positionOut].y;
            int blue = input[positionOut].z;

            if (x <= 0 && y >= 0){
                windowPosition = 0; // top left
            }

            if (x >= 0 && y >= 0){
                windowPosition = 1; //top right
            }

            if (x <= 0 && y <= 0){
                windowPosition = 2; // bottom left
            }

            if (x >= 0 && y <= 0){
                windowPosition = 3; // bottom right
            }

            meanRGB[windowPosition][0] += red;
            meanRGB[windowPosition][1] += green;
            meanRGB[windowPosition][2] += blue;

            window[windowPosition] += max(red, max(green,blue));
            pxCount[windowPosition]++;

            SD[windowPosition] += pow((max(red, max(green,blue)) -
                window[windowPosition]),2.0);
        }
    }

    for (int i = 0; i < 4; i++){
        SD[i] = sqrt(SD[i] / (pxCount[i])); 
        window[i] /= pxCount[i];
        for(int j = 0; j < 3; j++){ 
            meanRGB[i][j] /= pxCount[i];
        }  
    } 

    double minSD = min(SD[0], min( SD[1], min(SD[2], SD[3])));
    if (minSD == SD[0])  tidx=0;
    else if (minSD == SD[1]) tidx=1;
    else if (minSD == SD[2]) tidx=2;
    else tidx=3;
    
    output[tid].x = meanRGB[tidx][0];
    output[tid].y = meanRGB[tidx][1];
    output[tid].z = meanRGB[tidx][2]; 
}

void Labwork::labwork10_GPU(){
    // Calculate number of pixels
    int pixelCount = inputImage->width * inputImage->height;
    int windowSize = 32;

    // Allocate CUDA memory
    uchar3 *devInput;
    uchar3 *devOutput;
    cudaMalloc(&devInput, pixelCount *sizeof(uchar3));
    cudaMalloc(&devOutput, pixelCount *sizeof(uchar3));

    // Copy InputImage from CPU (host) to GPU (device)
    cudaMemcpy(devInput, inputImage->buffer, pixelCount * 
    sizeof(uchar3), cudaMemcpyHostToDevice);

    // Processing : launch the kernel
    // int blockSize = 1024;
    // int numBlock = pixelCount / blockSize;  
    // grayscale<<<numBlock, blockSize>>>(devInput, devOutput);
    dim3 blockSize = dim3(32, 32);
    //dim3 gridSize = dim3(8, 8);
    dim3 gridSize = dim3((inputImage->width + blockSize.x -1) / 
        blockSize.x, (inputImage->height + blockSize.y -1) / blockSize.y);
    kuwahara<<<gridSize, blockSize>>>(devInput, devOutput, 
        inputImage->width, inputImage->height, windowSize);
    // Copy CUDA Memory from GPU to CPU
    // allocate memory for the output on the host
    outputImage = static_cast<char *>(malloc(pixelCount * sizeof(uchar3)));  
    cudaMemcpy(outputImage, devOutput, pixelCount * 
        sizeof(uchar3), cudaMemcpyDeviceToHost);   

    // Cleaning
    cudaFree(devInput);
    cudaFree(devOutput);
}
\end{verbatim}

\end{enumerate}

\end{document}
        


