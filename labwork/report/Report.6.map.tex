\documentclass{article}
\usepackage[utf8]{inputenc}

\title{Report.6.map.tex}
\author{Kieu Quoc Viet }
\date{November 2019}

\begin{document}

\maketitle

\begin{enumerate}
    \item Explain how you implement the labworks
    \begin{itemize}
    \item Implement labwork 6a: grayscale image binarization
    \begin{verbatim}
__global__ void binarization(uchar3* input, uchar3* output, 
        int imageWidth, int imageHeight, int thresholdValue){
    int tidx = threadIdx.x + blockIdx.x * blockDim.x;
    int tidy = threadIdx.y + blockIdx.y * blockDim.y;
    if(tidx >= imageWidth || tidy >= imageHeight) return;
    int tid = tidx + tidy * imageWidth;
    
    unsigned char binary = (input[tid].x + input[tid].y + input[tid].z) / 3;
    if (binary > thresholdValue){
        binary = 255;
    } else {
        binary = 0;
    }
    output[tid].z = output[tid].y = output[tid].x = binary;
}        
    \end{verbatim}
    
    \item Implement labwork 6b: brightness control
    \begin{verbatim}
__global__ void brightness(uchar3* input, uchar3* output, 
        int imageWidth, int imageHeight, int brightnessValue){
    int tidx = threadIdx.x + blockIdx.x * blockDim.x;
    int tidy = threadIdx.y + blockIdx.y * blockDim.y;
    if(tidx >= imageWidth || tidy >= imageHeight) return;
    int tid = tidx + tidy * imageWidth;
    
    unsigned char binary = (input[tid].x + input[tid].y + input[tid].z) / 3;
    // unsigned char increase = binary + brightnessValue;
    // if (increase > 255){
    //     increase = 255;
    // } else {
    //     increase = 0;
    // }
    binary += brightnessValue;
    output[tid].z = output[tid].y = output[tid].x = binary;
}        
    \end{verbatim}
    
    \item Implement labwork 6c: blending two images
    \begin{verbatim}
__global__ void blending(uchar3* input0, uchar3* input1, uchar3* output, 
        int imageWidth, int imageHeight, float weightValue){
    int tidx = threadIdx.x + blockIdx.x * blockDim.x;
    int tidy = threadIdx.y + blockIdx.y * blockDim.y;
    if(tidx >= imageWidth || tidy >= imageHeight) return;
    int tid = tidx + tidy * imageWidth;

    // unsigned char binary = (input0[tid].x + input0[tid].y + input0[tid].z) / 3;
    // unsigned char binary2 = (input1[tid].x + input1[tid].y + input1[tid].z) / 3;
    // binary = weightValue*binary + (1-weightValue)*binary2;
    float binary = (input0[tid].x + input0[tid].y + input0[tid].z) / 3;
    float binary1 = (input1[tid].x + input1[tid].y + input1[tid].z) / 3;
    float totalbinary = (binary * weightValue) + binary1 * (1 - weightValue);

    output[tid].z = output[tid].y = output[tid].x = totalbinary; 
}        
    \end{verbatim}
    
    \item Implement the labwork
    \begin{verbatim}
void Labwork::labwork6_GPU() {

    /*6A - BINARIZATION
    int threshold;
    printf("Enter the threshold value: ");
    scanf("%d", &threshold);
    */

    /* 6B - BRIGHTNESS CONTROLL
    int bright;
    printf("Enter the threshold value: ");
    scanf("%d", &bright);
    */ 

    /* 6C - BLENDING
    char buffer[3];
    printf("Enter the weight: ", buffer);
    scanf("%s", buffer);
    int weightValue = atoi(buffer);
    */

    // Calculate number of pixels
    int pixelCount = inputImage->width * inputImage->height;

    // // Allocate CUDA memory
    uchar3 *devInput;
    uchar3 *devOutput;
    cudaMalloc(&devInput, pixelCount *sizeof(uchar3));
    cudaMalloc(&devOutput, pixelCount *sizeof(uchar3));

    // // Copy InputImage from CPU (host) to GPU (device)
    cudaMemcpy(devInput, inputImage->buffer, pixelCount * 
        sizeof(uchar3), cudaMemcpyHostToDevice);

    // // Processing : launch the kernel
    // // int blockSize = 1024;
    // // int numBlock = pixelCount / blockSize;  
    // // grayscale<<<numBlock, blockSize>>>(devInput, devOutput);
    dim3 blockSize = dim3(32, 32);
    // //dim3 gridSize = dim3(8, 8);
    dim3 gridSize = dim3((inputImage->width + blockSize.x -1) / 
        blockSize.x, (inputImage->height + blockSize.y -1) / blockSize.y);

    // 6A - BINARIZATION
    // binarization<<<gridSize, blockSize>>>(devInput, devOutput,
        inputImage->width, inputImage->height, threshold);
    // 6B - BRIGHTNESS CONTROLL
    //brightness<<<gridSize, blockSize>>>(devInput, devOutput, 
        inputImage->width, inputImage->height, bright);
    // 6C - BLENDING
    //blending<<<gridSize, blockSize>>>(devInput, devInput, devOutput,
        inputImage->width, inputImage->height, weightValue);

    // // Copy CUDA Memory from GPU to CPU
    // // allocate memory for the output on the host
    outputImage = static_cast<char *>(malloc(pixelCount * sizeof(uchar3)));  
    cudaMemcpy(outputImage, devOutput, pixelCount * 
        sizeof(uchar3), cudaMemcpyDeviceToHost);   

    // // Cleaning
    cudaFree(devInput);
    cudaFree(devOutput);
}        
    \end{verbatim}    
    \end{itemize}
    

\end{enumerate}

\end{document}
        



