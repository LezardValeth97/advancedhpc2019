\documentclass{article}
\usepackage[utf8]{inputenc}

\title{Report.0.investigate.tex}
\author{Kieu Quoc Viet }
\date{October 2019}

\begin{document}

\maketitle

\begin{enumerate}
    \item How you implement the conversion?
    Add \textbf{#pragma omp parallel for schedule(dynamic)} or \textbf{#pragma omp single}, etc before the loop.
    \begin{verbatim}
        #pragma omp master
   
        for (int j = 0; j < 100; j++) {     // let's do it 100 times, otherwise it's too fast!
            for (int i = 0; i < pixelCount; i++) {
            outputImage[i * 3] = (char) (((int) inputImage->buffer[i * 3] + (int) inputImage->buffer[i * 3 + 1] +
                                          (int) inputImage->buffer[i * 3 + 2]) / 3);
            outputImage[i * 3 + 1] = outputImage[i * 3];
            outputImage[i * 3 + 2] = outputImage[i * 3];
        }
    }
    \end{verbatim}
    
    \item{What’s the speedup?}
    The program's duration is faster than sequetial one using parallel computing.
    \begin{itemize}
        \item Using \textbf{#pragma omp single}
        \begin{verbatim}
            USTH ICT Master 2018, Advanced Programming for HPC.
            Warming up...
            Starting labwork 1
            labwork 1 CPU ellapsed 3363.8ms
            labwork 1 ellapsed 201.9ms
        \end{verbatim}
        
        \item Using \textbf{#pragma omp parallel for schedule(dynamic)}
        \begin{verbatim}
            USTH ICT Master 2018, Advanced Programming for HPC.
            Warming up...
            Starting labwork 1
            labwork 1 CPU ellapsed 3141.4ms
            labwork 1 ellapsed 49.0ms
        \end{verbatim}
        
        \item Using \textbf{#pragma omp master}
        \begin{verbatim}
        USTH ICT Master 2018, Advanced Programming for HPC.
        Warming up...
        Starting labwork 1
        labwork 1 CPU ellapsed 3167.9ms
        labwork 1 OpenMP ellapsed 2933.3ms
        labwork 1 ellapsed 3026.1ms
        \end{verbatim}
    \end{itemize}
\end{document}
