\documentclass{article}
\usepackage[utf8]{inputenc}

\title{Report.4.threads.tex}
\author{Kieu Quoc Viet }
\date{November 2019}

\begin{document}

\maketitle

\begin{enumerate}
    \item Explain how you improve the labwork
    
    \begin{verbatim}
// Write a grayscale kernel
__global__ void grayscaleVer2D(uchar3 *input, uchar3 *output, int width, 
    int height) {
    // this will execute in a device core
    int tidx = threadIdx.x + blockIdx.x * blockDim.x;
    if (tidx >= width) return;
    int tidy = threadIdx.y + blockIdx.y * blockDim.y;
    if (tidy >= height) return;

    int tid = tidx + tidy * width;
    output[tid].x = (input[tid].x + input[tid].y +input[tid].z) / 3;
    output[tid].z = output[tid].y = output[tid].x;
}

void Labwork::labwork4_GPU() {
    // Calculate number of pixels
    int pixelCount = inputImage->width * inputImage->height;
 
    // Allocate CUDA memory
    uchar3 *devInput;
    uchar3 *devOutput;
    cudaMalloc(&devInput, pixelCount *sizeof(uchar3));
    cudaMalloc(&devOutput, pixelCount *sizeof(uchar3));

    // Copy InputImage from CPU (host) to GPU (device)
    cudaMemcpy(devInput, inputImage->buffer, pixelCount 
        * sizeof(uchar3),cudaMemcpyHostToDevice);

    // Processing : launch the kernel
    // int blockSize = 1024;
    // int numBlock = pixelCount / blockSize;  
    // grayscale<<<numBlock, blockSize>>>(devInput, devOutput);
    dim3 blockSize = dim3(1024, 1024);
    //dim3 gridSize = dim3(8, 8);
    dim3 gridSize = dim3((inputImage->width + blockSize.x -1) / 
        blockSize.x, (inputImage->height + blockSize.y -1) / blockSize.y);
    grayscaleVer2D<<<gridSize, blockSize>>>(devInput, devOutput, 
        inputImage->width, inputImage->height);

    // Copy CUDA Memory from GPU to CPU
    // allocate memory for the output on the host
    outputImage = static_cast<char *>(malloc(pixelCount * sizeof(uchar3)));  
    cudaMemcpy(outputImage, devOutput, 
        pixelCount * sizeof(uchar3),cudaMemcpyDeviceToHost);   

    // Cleaning
    cudaFree(devInput);
    cudaFree(devOutput);
}
\end{verbatim}
    
    \item Compare speedup with previous 1D grid??\\
    Labwork 3 ellapsed 140.1ms\\
    Labwork 4 ellapsed 172.7ms\\
    It is slower than the 1D grid.

    \item Answer the questions in the upcoming slides, explain why?
    \begin{enumerate}
        \item Exercise 1
        \begin{enumerate}
            \item Consider a GPU having the following specs (maximum numbers):
            \begin{itemize}
                \item 512 threads/block
                \item 1024 threads/SM
                \item 8 blocks/SM
                \item 32 threads/warp
            \end{itemize}
            
            \item Question: What is the best configuration for thread blocks to implement grayscaling?
            \item Answer: 
        \end{enumerate}
        
        \item Exercise 2
        \begin{enumerate}
            \item Consider a device SM that can take max:
            \begin{itemize}
                \item 1536 threads
                \item 4 blocks
            \end{itemize}
            \item Question: Which of the following block configs would result in the mostnumber of threads in the SM??
            \item Answer: 512 threads/blk, when the SM takes 3 blocks, it runs 1,536 threads at full limit
        \end{enumerate}
        
        \item Exercise 3
        \begin{enumerate}
            \item Consider a vector addition problem
            \begin{itemize}
                \item Vector length is 2,000
                \item Each thread produces one output
                \item Block size 512 threads
                \item [] How many threads will be in the grid?
            \end{itemize}
            \item Answer: 4 blocks $\times$ 512 = 2048 threads.
        \end{enumerate}
        
    \end{enumerate}

\end{enumerate}

\end{document}
        


