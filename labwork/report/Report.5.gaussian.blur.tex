\documentclass{article}
\usepackage[utf8]{inputenc}

\title{Report.5.gaussian.blur.tex}
\author{Kieu Quoc Viet }
\date{November 2019}

\begin{document}

\maketitle

\begin{enumerate}
    \item Explain how you implement the labworks

    \begin{verbatim}

// write a blur kernel for shared memory
__global__ void blur(uchar3* input, uchar3* output, 
        int* kernel, int imageWidth, int imageHeight){
    int tidx = threadIdx.x + blockIdx.x * blockDim.x;
    int tidy = threadIdx.y + blockIdx.y * blockDim.y;
    if(tidx >= imageWidth || tidy >= imageHeight) return;
    int tid = tidx + tidy * imageWidth;

    __shared__ int sKernel[49];
    int localtid = threadIdx.x + threadIdx.y * blockDim.x;
    if (localtid < 49){
        sKernel[localtid] = kernel[localtid];
    }
    __syncthreads();

    int sum = 0; // sum is for normalization
    int constant = 0;
    for(int y=-3; y < 3; y++){
        for(int x=-3; x < 3; x++){
            int rows = tidx + x;
            int columns = tidy + y;
            if( rows < 0 || rows >= imageWidth || columns < 0 
                || columns >= imageHeight) continue;
            int tid = rows + columns * imageWidth;
            unsigned char pixelValue = (input[tid].x + input[tid].y +
                input[tid].z) / 3;
            int coefficient = sKernel[(y+3)*7+x+3];
            sum += pixelValue*coefficient;
            constant += coefficient;
        }
    }
    sum /= constant;
    // int positionOut = y*inputImage->width + x;
    // if(positionOut < pixelCount){
    //     outputImage[positionOut * 3] = outputImage[positionOut * 3 + 1] = outputImage[positionOut * 3 + 2] = sum;
    // }
    output[tid].z = output[tid].y = output[tid].x = sum;
}

void Labwork::labwork5_GPU() {

    int kernel[] = {0,0,1,2,1,0,0,
                    0,3,13,22,13,3,0,
                    1,13,59,97,59,13,1,
                    2,22,97,159,97,22,2,
                    1,13,59,97,59,13,1,
                    0,3,13,22,13,3,0,
                    0,0,1,2,1,0,0};
    int *share;
    // Calculate number of pixels
    int pixelCount = inputImage->width * inputImage->height;
    
    dim3 blockSize = dim3(32, 32);
    //dim3 gridSize = dim3(8, 8);
    dim3 gridSize = dim3((inputImage->width + blockSize.x -1) / 
        blockSize.x, (inputImage->height + blockSize.y -1) / blockSize.y);

    // Allocate CUDA memory
    uchar3 *devInput;
    uchar3 *devOutput;
    cudaMalloc(&devInput, pixelCount *sizeof(uchar3));
    cudaMalloc(&devOutput, pixelCount *sizeof(uchar3));
    cudaMalloc(&share, sizeof(kernel));
    // allocate memory for the output on the host
    outputImage = static_cast<char *>(malloc(pixelCount * sizeof(uchar3)));  

    // Copy InputImage from CPU (host) to GPU (device)
    cudaMemcpy(devInput, inputImage->buffer, pixelCount * 
        sizeof(uchar3), cudaMemcpyHostToDevice);

    // Copy Kernel into shared memory
    cudaMemcpy(share, kernel, sizeof(kernel), cudaMemcpyHostToDevice);

    // Processing : launch the kernel
    // int blockSize = 1024;
    // int numBlock = pixelCount / blockSize;  
    blur<<<gridSize, blockSize>>>(devInput, devOutput, share, 
        inputImage->width, inputImage->height);

    // Copy CUDA Memory from GPU to CPU
    cudaMemcpy(outputImage, devOutput, pixelCount * 
        sizeof(uchar3), cudaMemcpyDeviceToHost);   

    // // Cleaning
    cudaFree(devInput);
    cudaFree(devOutput);
    cudaFree(share);
}
\end{verbatim}

\end{enumerate}

\end{document}
        


